\section{Интерполяция}
\paragraph{1.} Имеется набор значений функции
%TODO: таблица
где можно считать, что $ x_0 < x_1 < \ldots < x_n $ (во всяком случае, все они
попарно различны). Тогда существует единственный многочлен $ f $ степени $ \deg
f\leqslant n $, такой что $ f(x_i) = y_i $, где $ i = 0, \ldots, n $.

Запишем многочлен в виде  
\begin{gather*}
  f= a_0 + a_1 x_ + \ldots + a_n x^n,\\
   \begin{cases}
    a_0 + a_1x_0 + \ldots + a_n x^n_0 = y_0,\\
    \ldots,\\
    a_0 + a_1x_n + \ldots + a_nx^n_n = y_n.
  \end{cases}
\end{gather*}
Здесь $ (n+1) $ уравнение и столько же переменных. Определитель СЛАУ есть
определитель Вандермонда, он не равен нулю, поскольку все точки $ x_i $ попарно
различны. Поэтому система имеет и единственное решение. Эту систему можно просто
решить. Можно поступить хитрее.

Второй способ. 
%TODO: n таблиц

По теореме Безу $ g_0 $ $ x_1, \ldots, x_n $ --- корни, откуда $ g\vdots
(x-x_1), \ldots, (x-x_n) $. $ g_0(x) = c_0(x-x_1) \cdot \ldots \cdot (x-x_n) $,
$ c_0 \in \mathbb R $. $ 1 = g_0(x_0) = c_0(x_0 - x_1) \cdot\ldots\cdot(x_0 -
x_n) $, поэтому  
\[
    %TODO
    g_0(x) = \frac{(x-x_1)\ldots(x-x_n)}{(x_0-x_1)\ldots(x_0-x_n)}=
    \frac{\prod_{i=1}^n(x-x_i)}{\prod_{i=1}^n(x_0 - x_1)}.
\]
Тогда  
\begin{align*}
  f(x) &= y_0g_0(x) + y_1g_1(x) + \ldots + y_ng_n(x),\\
  f(x_0)&= y_0 g_0(x_0) + y_1g_1(x_0) + \ldots + y_ng_n(x_0)\ldots,
\end{align*}
где на второй строчке $ g_0(x_0) = 1 $, $ g_1(x_0) = 0 $, $ g_n(x_0) = 0 $.

\paragraph{Интерполяционный полином Лагранжа.} Получили формулу  
\[
  f(x) = \sum_{i=0}^n y_i \frac{\prod\limits_{j=0, \ldots, n; j\neq i}
  (x-x_j)}{\prod\limits_{j=0, \ldots, n; j\neq i} (x_i - x_j)}
\]

Третий способ.
%TODO: таблицы

 
\begin{align*}
  f_0(x) &= y_0,\\
  (f_1-f_0)(x_0) &= 0 \implies (x-x_0) \mid f_1-f_0 \implies = q_1(x-x_0),\\
  f_1&= f_0 + q_1(x-x_0), \quad y_1 = f_1(x_1) = f_0(y_1) + q_1(x_1-x_0),\\
  q_1 &= \frac{y_1 - y_0}{x_1 -x_0}, \qquad f_1 = y_0 + \frac{y_1
  -y_1}{x_1-x_0}(x-x_0),\\
    f_i - f_{i-1} &= q_i(x-x_0)\ldots (x-x_{i-1}),\\
    y_i &= f_i(x_i) = f_{i-1}(x_i) + q_i(x_i-x_0) \ldots (x_i - x_{i-1}),\\
    a_i &= \frac{y_i - f_{i-1}(x_i)}{(x_i-x_0)\ldots(x_i-x_{i-1})}.
\end{align*}
Интерполяция помогает и удобный инструмент, когда точек мало. Когда точек много,
точная интерполяция сложна и бессмысленна.

\paragraph{2. Приближённая интерполяция.} Когда точек много и точки известны
приблизительно, используют приближённую интерполяцию. \emph{Метод наименьших
квадратов}.
%TODO: рисунок, таблица

Ищем многочлен $ f $ степени $ \deg f = m < n $. Так называемая \emph{<<невязка>>}
$ y_k - f(x_k) $. Как минимизировать невязки? Самый популярный способ ---
минимизировать сумму квадратов невязок. Тогда  
\[
  \Phi(a_0, \ldots, a_m) = \sum_{k=0}^n (a_0 + a_1 x_k + \ldots + a_m x_k^m -
  y_k)^2 \to \min.
\]
Максимума нет. Минимум означает, что $ \frac{\partial \Phi}{\partial a_0} = 0 $,
$ \ldots $, $ \frac{\partial \Phi}{\partial a_m} = 0 $.

