\section{Одномерная оптимизация без использования производной}
Часто случается, что производную сложно или невозможно найти, поскольку,
например, сама имеет неизвестное уравнение.

Попробуем найти экстремумы без производной, даже не дифференцируемой.

\begin{definition}
  Функция $ f(x)\colon [a,b] \to \mathbb R $ называется \emph{унимодальной},
  если она непрерывна на этом отрезке и существуют такие числа $ a \leqslant \alpha
  \leqslant \beta \leqslant b$, что
  \begin{itemize}[label=---]
    % \item $0 \leqslant \alpha \leqslant \beta \leqslant b$,\\
    \item функция $ f(x) $ строго монотонно убывает на $ [a, \alpha] $,\\
    \item $ f(x) \equiv c $ для всех $ x \in [\alpha, \beta] $,\\
    \item $ f(x) $ строго монотонно возрастает на $ [\beta, b] $.      
  \end{enumerate}
  
Функция называется \emph{строго унимодальной}, если $ \alpha = \beta $.
\end{definition}

\begin{remark}
  Выпуклая вниз непрерывная на $ [a, b] $ функция унимодальна на этом отрезке.
\end{remark}
\begin{remark}
  У унимодальной функции $ \Omega_\ast = [\alpha, \beta] $ минимум $ m_\ast = c
  $. Множество точек минимума.
\end{remark}
\begin{remark}
  Есно, что если функция $ f(x) $ унимодальна на $ [a, b] $ и $ [c,d] \subset
  [a,b] $, то $ f(x) $ унимодальна на $ [c, d] $.
\end{remark}
\setcounter{remark}{0}

\paragraph{1. Метод деления отрезка пополам.} Пусть $ f(x) $ унимодальна на $
[a,b] $, где $ b > a $. Выберем параметр $ 0 < \delta < b-a $. Выберем две
точки 
\[
    u_1 = \frac{a + b -\delta}{2}, \quad u_2 = \frac{a+ b+\delta}{2}.
\]
Вычислим $ f(u_1) $, $ f(u_2) $.
\begin{enumerate}
  \item[1 сл.] $ f(u_1) \leqslant f(u_2) $. Тогда полагаем $ a_2 = a_1$,$b_2=u_2
    $,
  \item[2 сл.] $ f(u_1) > f(u_2) $. Тогда полагаем $ a_2 = u_1 $, $ b_2 = b_1 $.
\end{enumerate}
Тогда $ f $ унимодальна на $ [a_2, b_2] $.

Вычисляем $ f(u_{2k-1}) $ и $ f(u_{2k}) $.
\begin{enumerate}
  \item[1 сл.] $ f(u_{2k-1}) \leqslant f(u_{2k}) $. Тогда $ a_{k+1} = a_k $, $ b_{k+1} = u_{2k} $,
  \item[2 сл.] $ f(u_{2k-1}) > f(u_{2k}) $. $ a_{k+1} = u_{2k-1} $, $ b_{k+1} = b_k $.
\end{enumerate}
$ l([a_{k+1}, b_{k+1}]) = \frac{b-a -\delta}{2^{k+1}} + \delta > \delta$.

%TODO: пропуск
