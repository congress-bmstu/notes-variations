\section{Основные понятия}
% \paragraph{Основные примеры простейших задач.} 
\paragraph{Терминология.}
\begin{enumerate}
  \item Целевая функция --- функция, у
которой мы ищем экстремум.
\item Объект оптимизации --- то, что мы оптимизируем.
\item Параметры оптимизации --- параметры, от которых зависит целевая функция
  (аргументы).
\item Ограничения --- равенства и неравенства, наложенные на параметры.
\end{enumerate}

\begin{example}
  Спроектировать бак горючего в виде прямого кругового цилиндра с заданным
  объёмом $ V $, на изготовление которого будет потрачено наименьшее количество
  листовой стали, то есть он должен иметь \emph{наименьшую площадь поверхности}.
  \begin{solution}
    %TODO: рисунок цилиндра!
    Параметры оптимизации: $ R $ и $ H $. Целевая функция --- функция площади
    поверхности, $ S = 2\pi R^2 + 2\pi RH = 2\pi R(R + H) \to \min $.
    Ограничения: $ \pi R^2 H = V $ ($ V > 0 $); $ R $, $ H > 0 $. По известной
    формуле
    \[
           H &= \frac{V}{\pi R^2}.
         \]
         Тогда
   \begin{multline*}
     S = 2\pi R \left( \frac{V}{\pi R^2} + R\right) = \frac{2\pi R V}{\pi R^2} + 2\pi R^2
     =\\= \frac{2 V}{R} + 2\pi R^2 = \frac{V}{R} + \frac{V}{R} + 2\pi R^2
     \geqslant
        3 \sqrt[3]{\frac{V}{R}\frac{V}{R}2\pi R^2} = 3 \sqrt[3]{2\pi
       V^2}. = S_{\min}.
   \end{multline*}
   %TODO: пояснить
   Здесь было использовано неравенство между средним арифметическим и средним
   геометрическим. Есть способ проще (через производные). Тогда $ S = S_{\min} $ при $ V/R =
   2\pi R^2 $, то есть $ V = 2\pi R^3 $ и $ R = \sqrt[3]{\frac{V}{2\pi}} $, а  
   \[
     H = \frac{V}{\pi R^2} = \frac{V \sqrt[3]{4\pi^2}}{\pi \sqrt[3]{V^2}} =
     \sqrt[3]{\frac{4V}{\pi}}.
   \]
   Отсюда  
   \[
     2R = 2\sqrt[3]{\frac{V}{2\pi}} = \sqrt[3]{\frac{8^4 V}{2\pi}}=H.
     %TODO: поправить предпоследнее
   \]
  \end{solution}
\end{example}

  Линейное программирование придумали в России (см. Канторович).

  \begin{example}[линейное программирование]
  Предприятие выпускает $ m $ наименований продукции, используя $ n $ видов
  ресурсов. Обозначим $ a_{ij} $ --- затраты $ i $-го вида ресурса на
  производство $ j $-го вида продукции; $ x_j $ --- планируемый объём выпуска
  продукции; $ d_j $ --- цена единицы продукции $ j $-го вида; а $ (x_1, \ldots,
  x_n) $ --- оптимальный план. Через $ b_i $ обозначим общий запас ресурса вида
  $ i $.

  Тогда целевая функция имеет вид
  \[
    S = \sum_{j=1}^n d_j x_j \to \max,
  \]
  а ограничения будут следующими: $ a_j \leqslant x_j \leqslant A_j $; $
  a_{i1}x_1 + \ldots + a_{in}x_n \leqslant b_i $. Нишевые продукты тоже должны
  производиться. Аналогично, в транспортной
  задаче, задаче коммивояжёра и пр.
  
\end{example}


