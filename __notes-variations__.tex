\documentclass[12pt]{article}
\usepackage{amsthm, mathrsfs, mathtools, amssymb}
\usepackage[T2A]{fontenc}
\usepackage[utf8]{inputenc}
\usepackage{xcolor}

\usepackage[english,russian]{babel}
\usepackage[colorlinks = true, linkcolor = black, urlcolor = black]{hyperref}
\usepackage[depth = subsection]{bookmark}
\usepackage{enumitem}

\ifx\pdfoutput\undefined
\usepackage{graphicx}
\else
\usepackage[pdftex]{graphicx}
\fi
\usepackage{wrapfig}

 %\DeclareMathOperator{\pr}{pr}
 %\DeclareMathOperator{\arctg}{arctg}
 %\DeclareMathOperator{\arcctg}{arcctg}
 %\DeclareMathOperator{\ch}{ch}
 %\DeclareMathOperator{\sh}{sh}

\usepackage{bm}


\newtheoremstyle{example}% name
{0.7cm}% Space above
{0.7cm}% Space below
{\small}% Body font
{}% Indent amount
{\small\scshape}% Theorem head font
{.}% Punctuation after theorem head
{.5em}% Space after theorem head
{}% Theorem head spec (can be left empty, meaning ‘normal’)

\newtheoremstyle{algorithm}
{0.7cm}
{0.7cm}
{}
{}
{\tt}
{.}
{.5em}
{}

\theoremstyle{algorithm}
\newtheorem{alg}{Алгоритм}
\newtheorem{alg*}{Алгоритм}

\theoremstyle{example}
\newtheorem{example}{Пример}

\theoremstyle{plain}
\newtheorem{theorem}{Теорема}
\newtheorem{corollary}{Следствие}
\newtheorem*{corollary*}{Следствие} 
\newtheorem{lemma}{Лемма}
\newtheorem{utv}{Утверждение}
\newtheorem*{utv*}{Утверждение}

\theoremstyle{definition}
\newtheorem{definition}{Определение}
\newtheorem*{definition*}{Определение}
\newtheorem{question}{Вопрос}

\theoremstyle{remark}
\newtheorem{remark}{Замечание}
\newtheorem*{remark*}{Замечание}
\numberwithin{remark}{section}

\frenchspacing

\usepackage[labelsep=period]{caption}
\captionsetup{font = small}

\newcommand{\Hom}{\mathrm{Hom}}
\newcommand{\Spl}{\mathrm{Spl}}
\newcommand{\spl}{\mathrm{spl}}

\usepackage{bop}
\graphicspath{
    {images00-01/}{images00-02/}
      {images01-01/}{images01-02/}{images01-03/}
}
\usepackage{showlabels}
% будет показывать ссылки прямо в pdf. когда файл готов, лучше убрать

\usepackage{geometry}
\geometry{verbose,a4paper,tmargin=1cm,bmargin=2cm,lmargin=1.5cm,rmargin=1.5cm}



\begin{document}
\title{Методы оптимизации}
\maketitle
\tableofcontents

\newpage
\paragraph{Литература.}
\begin{enumerate}
  \item \emph{Васильев --- Численные методы решения экстремальных задач}.
    Функциональный анализ, сжато, сложно, математично.
\end{enumerate}

\section{Основные понятия}
% \paragraph{Основные примеры простейших задач.} 
\paragraph{Терминология.}
\begin{enumerate}
  \item Целевая функция --- функция, у
которой мы ищем экстремум.
\item Объект оптимизации --- то, что мы оптимизируем.
\item Параметры оптимизации --- параметры, от которых зависит целевая функция
  (аргументы).
\item Ограничения --- равенства и неравенства, наложенные на параметры.
\end{enumerate}

\begin{example}
  Спроектировать бак горючего в виде прямого кругового цилиндра с заданным
  объёмом $ V $, на изготовление которого будет потрачено наименьшее количество
  листовой стали, то есть он должен иметь \emph{наименьшую площадь поверхности}.
  \begin{solution}
    %TODO: рисунок цилиндра!
    Параметры оптимизации: $ R $ и $ H $. Целевая функция --- функция площади
    поверхности, $ S = 2\pi R^2 + 2\pi RH = 2\pi R(R + H) \to \min $.
    Ограничения: $ \pi R^2 H = V $ ($ V > 0 $); $ R $, $ H > 0 $. По известной
    формуле
    \[
           H &= \frac{V}{\pi R^2}.
         \]
         Тогда
   \begin{multline*}
     S = 2\pi R \left( \frac{V}{\pi R^2} + R\right) = \frac{2\pi R V}{\pi R^2} + 2\pi R^2
     =\\= \frac{2 V}{R} + 2\pi R^2 = \frac{V}{R} + \frac{V}{R} + 2\pi R^2
     \geqslant
        3 \sqrt[3]{\frac{V}{R}\frac{V}{R}2\pi R^2} = 3 \sqrt[3]{2\pi
       V^2}. = S_{\min}.
   \end{multline*}
   %TODO: пояснить
   Здесь было использовано неравенство между средним арифметическим и средним
   геометрическим. Есть способ проще (через производные). Тогда $ S = S_{\min} $ при $ V/R =
   2\pi R^2 $, то есть $ V = 2\pi R^3 $ и $ R = \sqrt[3]{\frac{V}{2\pi}} $, а  
   \[
     H = \frac{V}{\pi R^2} = \frac{V \sqrt[3]{4\pi^2}}{\pi \sqrt[3]{V^2}} =
     \sqrt[3]{\frac{4V}{\pi}}.
   \]
   Отсюда  
   \[
     2R = 2\sqrt[3]{\frac{V}{2\pi}} = \sqrt[3]{\frac{8^4 V}{2\pi}}=H.
     %TODO: поправить предпоследнее
   \]
  \end{solution}
\end{example}

  Линейное программирование придумали в России (см. Канторович).

  \begin{example}[линейное программирование]
  Предприятие выпускает $ m $ наименований продукции, используя $ n $ видов
  ресурсов. Обозначим $ a_{ij} $ --- затраты $ i $-го вида ресурса на
  производство $ j $-го вида продукции; $ x_j $ --- планируемый объём выпуска
  продукции; $ d_j $ --- цена единицы продукции $ j $-го вида; а $ (x_1, \ldots,
  x_n) $ --- оптимальный план. Через $ b_i $ обозначим общий запас ресурса вида
  $ i $.

  Тогда целевая функция имеет вид
  \[
    S = \sum_{j=1}^n d_j x_j \to \max,
  \]
  а ограничения будут следующими: $ a_j \leqslant x_j \leqslant A_j $; $
  a_{i1}x_1 + \ldots + a_{in}x_n \leqslant b_i $. Нишевые продукты тоже должны
  производиться. Аналогично, в транспортной
  задаче, задаче коммивояжёра и пр.
  
\end{example}



\section{Интерполяция}
\paragraph{1.} Имеется набор значений функции
%TODO: таблица
где можно считать, что $ x_0 < x_1 < \ldots < x_n $ (во всяком случае, все они
попарно различны). Тогда существует единственный многочлен $ f $ степени $ \deg
f\leqslant n $, такой что $ f(x_i) = y_i $, где $ i = 0, \ldots, n $.

Запишем многочлен в виде  
\begin{gather*}
  f= a_0 + a_1 x_ + \ldots + a_n x^n,\\
   \begin{cases}
    a_0 + a_1x_0 + \ldots + a_n x^n_0 = y_0,\\
    \ldots,\\
    a_0 + a_1x_n + \ldots + a_nx^n_n = y_n.
  \end{cases}
\end{gather*}
Здесь $ (n+1) $ уравнение и столько же переменных. Определитель СЛАУ есть
определитель Вандермонда, он не равен нулю, поскольку все точки $ x_i $ попарно
различны. Поэтому система имеет и единственное решение. Эту систему можно просто
решить. Можно поступить хитрее.

Второй способ. 
%TODO: n таблиц

По теореме Безу $ g_0 $ $ x_1, \ldots, x_n $ --- корни, откуда $ g\vdots
(x-x_1), \ldots, (x-x_n) $. $ g_0(x) = c_0(x-x_1) \cdot \ldots \cdot (x-x_n) $,
$ c_0 \in \mathbb R $. $ 1 = g_0(x_0) = c_0(x_0 - x_1) \cdot\ldots\cdot(x_0 -
x_n) $, поэтому  
\[
    %TODO
    g_0(x) = \frac{(x-x_1)\ldots(x-x_n)}{(x_0-x_1)\ldots(x_0-x_n)}=
    \frac{\prod_{i=1}^n(x-x_i)}{\prod_{i=1}^n(x_0 - x_1)}.
\]
Тогда  
\begin{align*}
  f(x) &= y_0g_0(x) + y_1g_1(x) + \ldots + y_ng_n(x),\\
  f(x_0)&= y_0 g_0(x_0) + y_1g_1(x_0) + \ldots + y_ng_n(x_0)\ldots,
\end{align*}
где на второй строчке $ g_0(x_0) = 1 $, $ g_1(x_0) = 0 $, $ g_n(x_0) = 0 $.

\paragraph{Интерполяционный полином Лагранжа.} Получили формулу  
\[
  f(x) = \sum_{i=0}^n y_i \frac{\prod\limits_{j=0, \ldots, n; j\neq i}
  (x-x_j)}{\prod\limits_{j=0, \ldots, n; j\neq i} (x_i - x_j)}
\]

Третий способ.
%TODO: таблицы

 
\begin{align*}
  f_0(x) &= y_0,\\
  (f_1-f_0)(x_0) &= 0 \implies (x-x_0) \mid f_1-f_0 \implies = q_1(x-x_0),\\
  f_1&= f_0 + q_1(x-x_0), \quad y_1 = f_1(x_1) = f_0(y_1) + q_1(x_1-x_0),\\
  q_1 &= \frac{y_1 - y_0}{x_1 -x_0}, \qquad f_1 = y_0 + \frac{y_1
  -y_1}{x_1-x_0}(x-x_0),\\
    f_i - f_{i-1} &= q_i(x-x_0)\ldots (x-x_{i-1}),\\
    y_i &= f_i(x_i) = f_{i-1}(x_i) + q_i(x_i-x_0) \ldots (x_i - x_{i-1}),\\
    a_i &= \frac{y_i - f_{i-1}(x_i)}{(x_i-x_0)\ldots(x_i-x_{i-1})}.
\end{align*}
Интерполяция помогает и удобный инструмент, когда точек мало. Когда точек много,
точная интерполяция сложна и бессмысленна.

\paragraph{2. Приближённая интерполяция.} Когда точек много и точки известны
приблизительно, используют приближённую интерполяцию. \emph{Метод наименьших
квадратов}.
%TODO: рисунок, таблица

Ищем многочлен $ f $ степени $ \deg f = m < n $. Так называемая \emph{<<невязка>>}
$ y_k - f(x_k) $. Как минимизировать невязки? Самый популярный способ ---
минимизировать сумму квадратов невязок. Тогда  
\[
  \Phi(a_0, \ldots, a_m) = \sum_{k=0}^n (a_0 + a_1 x_k + \ldots + a_m x_k^m -
  y_k)^2 \to \min.
\]
Максимума нет. Минимум означает, что $ \frac{\partial \Phi}{\partial a_0} = 0 $,
$ \ldots $, $ \frac{\partial \Phi}{\partial a_m} = 0 $.


\section{Одномерная оптимизация без использования производной}
Часто случается, что производную сложно или невозможно найти, поскольку,
например, сама имеет неизвестное уравнение.

Попробуем найти экстремумы без производной, даже не дифференцируемой.

\begin{definition}
  Функция $ f(x)\colon [a,b] \to \mathbb R $ называется \emph{унимодальной},
  если она непрерывна на этом отрезке и существуют такие числа $ a \leqslant \alpha
  \leqslant \beta \leqslant b$, что
  \begin{itemize}[label=---]
    % \item $0 \leqslant \alpha \leqslant \beta \leqslant b$,\\
    \item функция $ f(x) $ строго монотонно убывает на $ [a, \alpha] $,\\
    \item $ f(x) \equiv c $ для всех $ x \in [\alpha, \beta] $,\\
    \item $ f(x) $ строго монотонно возрастает на $ [\beta, b] $.      
  \end{enumerate}
  
Функция называется \emph{строго унимодальной}, если $ \alpha = \beta $.
\end{definition}

\begin{remark}
  Выпуклая вниз непрерывная на $ [a, b] $ функция унимодальна на этом отрезке.
\end{remark}
\begin{remark}
  У унимодальной функции $ \Omega_\ast = [\alpha, \beta] $ минимум $ m_\ast = c
  $. Множество точек минимума.
\end{remark}
\begin{remark}
  Есно, что если функция $ f(x) $ унимодальна на $ [a, b] $ и $ [c,d] \subset
  [a,b] $, то $ f(x) $ унимодальна на $ [c, d] $.
\end{remark}
\setcounter{remark}{0}

\paragraph{1. Метод деления отрезка пополам.} Пусть $ f(x) $ унимодальна на $
[a,b] $, где $ b > a $. Выберем параметр $ 0 < \delta < b-a $. Выберем две
точки 
\[
    u_1 = \frac{a + b -\delta}{2}, \quad u_2 = \frac{a+ b+\delta}{2}.
\]
Вычислим $ f(u_1) $, $ f(u_2) $.
\begin{enumerate}
  \item[1 сл.] $ f(u_1) \leqslant f(u_2) $. Тогда полагаем $ a_2 = a_1$,$b_2=u_2
    $,
  \item[2 сл.] $ f(u_1) > f(u_2) $. Тогда полагаем $ a_2 = u_1 $, $ b_2 = b_1 $.
\end{enumerate}
Тогда $ f $ унимодальна на $ [a_2, b_2] $.

Вычисляем $ f(u_{2k-1}) $ и $ f(u_{2k}) $.
\begin{enumerate}
  \item[1 сл.] $ f(u_{2k-1}) \leqslant f(u_{2k}) $. Тогда $ a_{k+1} = a_k $, $ b_{k+1} = u_{2k} $,
  \item[2 сл.] $ f(u_{2k-1}) > f(u_{2k}) $. $ a_{k+1} = u_{2k-1} $, $ b_{k+1} = b_k $.
\end{enumerate}
$ l([a_{k+1}, b_{k+1}]) = \frac{b-a -\delta}{2^{k+1}} + \delta > \delta$.

%TODO: пропуск

\section{Одномерная оптимизация с использованием дифференцирования}
Пусть функция дифференцируема и мы знаем её производную. Однако по какой-то
причине уравнение $ f'(x) = 0 $ мы решить не в состоянии.
\subsection{Метод средней точки}

Пусть функция непрерывно дифференцируемая на $ [a,b] $ и имеет единственную
стационарную точку (где $ f'(x) = 0 $). Пусть также $ f'(a) <0 $, $ f'(b) > 0
$\footnote{Производная известна в том смысле, что мы можем прикинуть её знак в
точках.}. Тогда стационарная точка --- точка минимума (очевидно).

Положим $ a_0 = a $, $ b_0 = b $. Возьмём сеередину отрезка $ [a, b] $,  
\[
    c_0 = \frac{a+b}{2}.
\]
Вычислим $ f'(c_0) $. Далее как обычно в методе отрезке пополам.


\begin{alg}[Метод хорд]
  \begin{remark*} Пусть $ g(x) $ определена в точках $ x $, $ x_1 $, $ x_2 $, $
    x_1 \neq x_2 $ и $ g(x_1)\cdot g(x_2) < 0 $. Тогда прямая, проходящая через
    точки $ (x_1, g(x_1)) $ и $ (x_2, g(x_2)) $ имеет уравнение  
    \[
        y = \frac{g(x_2) - g(x_1)}{x_2 - x_1} x + \frac{g(x_1)x_2 -
        g(x_2)x_1}{x_2 - x_1},
    \]
   а точка пересечения этой прямой с осью абсцисс  
   \[
       x = \frac{g(x_2)x_1 - g(x_1)x_2}{g(x_2) - g(x_1)} = x_1 -
       \frac{g(x_1)(x_2-x_1)}{g(x_2) - g(x_1)} \in (x_1, x_2).
   \]
 \end{remark*}

 Пусть $ f(x) $ непрерывно дифференцируема на $ [a,b] $, имеет единственную
 стационарную точку на $ [a, b] $; $ f'(a) < 0 $, $ f'(b) > 0 $.

 Положим $ a_0 = a $, $ b_0 = b $.

 \texttt{$ (k+1) $-й шаг.} Рассматривается отрезок $ [a_k, b_k] $, $ f'(a_k) < 0
 $, $ f'(b_k) > 0 $. 

 Проведём прямую через точки $ (a_k, f'(a_k)) $, $ (b_k, f'(b_k)) $ и обозначим
 через 
 \[
     c_k = \frac{f'(b_k)a_k - f'(a_k)b_k}{f'(b_k) - f'(a_k)} \in (a_k, b_k).
 \]
Вычислим $ f'(c_k) $.
\begin{enumerate}[label=(\roman*).]
  \item $f'(c_k) = 0$. Тогда $ c_k  $ --- стационарная точка.
  \item $ f'(c_k) > 0 $. Тогда $ a_{k+1} = a_k $, $ b_{k+1} = c_k $.
  \item $ f'(c_k) < 0 $. Тогда $ a_{k+1} = c_k $, $ b_{k+1} = b_k $.
    %TODO: дописать
\end{enumerate}

\texttt{Условие остановки алгоритма.}
\begin{enumerate}[label=(\alph*)]
  \item Нашли стационарную точку (вряд ли),
  \item $ l([a_k, b_k]) < \varepsilon > 0$, где $ \varepsilon $ задан заранееб
  \item Выполнено заданное количество итераций.
\end{enumerate}

\begin{quotation}
  \emph{<<Понятно, что метод сходится>>.}
  \flushbottom --- Киреева.
\end{quotation}
\end{alg}

\begin{alg}[метод Ньютона, метод касательных]
  Пусть $ f(x) \in C^2[a,b] $, имеет единственную стационарную точку и $ f'(a) <
  0$, $ f'(b) > 0 $.

  Выберем начальную точку $ x_0 $.
  \begin{enumerate}[label=(\roman*)]
    \item $f'(x_0) = 0$. Тогда $ x_0 $ стационарная.
    \item $f'(x_0) \neq 0$. Тогда проведём касательную.
  \end{enumerate}
  
\end{alg}

\section{???}
\begin{alg}
  \texttt{Идея методов.} Пусть $ u_0 $ --- начальная точка, а  
  \[
    u_{k+1} = \ldots.
  \]
\begin{enumerate}[label=\roman*)]
  \item Если $ f'(u_k) = 0 $, то это стационарная точка,
  \item $ f'(u_k) \neq 0 $, то можно выбрать $ u_{k+1} $ так, что $ f(u_{k+1}) <
    f(u_k)$.
\end{alg}


\end{document}
