\section{Одномерная оптимизация с использованием дифференцирования}
Пусть функция дифференцируема и мы знаем её производную. Однако по какой-то
причине уравнение $ f'(x) = 0 $ мы решить не в состоянии.
\subsection{Метод средней точки}

Пусть функция непрерывно дифференцируемая на $ [a,b] $ и имеет единственную
стационарную точку (где $ f'(x) = 0 $). Пусть также $ f'(a) <0 $, $ f'(b) > 0
$\footnote{Производная известна в том смысле, что мы можем прикинуть её знак в
точках.}. Тогда стационарная точка --- точка минимума (очевидно).

Положим $ a_0 = a $, $ b_0 = b $. Возьмём сеередину отрезка $ [a, b] $,  
\[
    c_0 = \frac{a+b}{2}.
\]
Вычислим $ f'(c_0) $. Далее как обычно в методе отрезке пополам.


\begin{alg}[Метод хорд]
  \begin{remark*} Пусть $ g(x) $ определена в точках $ x $, $ x_1 $, $ x_2 $, $
    x_1 \neq x_2 $ и $ g(x_1)\cdot g(x_2) < 0 $. Тогда прямая, проходящая через
    точки $ (x_1, g(x_1)) $ и $ (x_2, g(x_2)) $ имеет уравнение  
    \[
        y = \frac{g(x_2) - g(x_1)}{x_2 - x_1} x + \frac{g(x_1)x_2 -
        g(x_2)x_1}{x_2 - x_1},
    \]
   а точка пересечения этой прямой с осью абсцисс  
   \[
       x = \frac{g(x_2)x_1 - g(x_1)x_2}{g(x_2) - g(x_1)} = x_1 -
       \frac{g(x_1)(x_2-x_1)}{g(x_2) - g(x_1)} \in (x_1, x_2).
   \]
 \end{remark*}

 Пусть $ f(x) $ непрерывно дифференцируема на $ [a,b] $, имеет единственную
 стационарную точку на $ [a, b] $; $ f'(a) < 0 $, $ f'(b) > 0 $.

 Положим $ a_0 = a $, $ b_0 = b $.

 \texttt{$ (k+1) $-й шаг.} Рассматривается отрезок $ [a_k, b_k] $, $ f'(a_k) < 0
 $, $ f'(b_k) > 0 $. 

 Проведём прямую через точки $ (a_k, f'(a_k)) $, $ (b_k, f'(b_k)) $ и обозначим
 через 
 \[
     c_k = \frac{f'(b_k)a_k - f'(a_k)b_k}{f'(b_k) - f'(a_k)} \in (a_k, b_k).
 \]
Вычислим $ f'(c_k) $.
\begin{enumerate}[label=(\roman*).]
  \item $f'(c_k) = 0$. Тогда $ c_k  $ --- стационарная точка.
  \item $ f'(c_k) > 0 $. Тогда $ a_{k+1} = a_k $, $ b_{k+1} = c_k $.
  \item $ f'(c_k) < 0 $. Тогда $ a_{k+1} = c_k $, $ b_{k+1} = b_k $.
    %TODO: дописать
\end{enumerate}

\texttt{Условие остановки алгоритма.}
\begin{enumerate}[label=(\alph*)]
  \item Нашли стационарную точку (вряд ли),
  \item $ l([a_k, b_k]) < \varepsilon > 0$, где $ \varepsilon $ задан заранееб
  \item Выполнено заданное количество итераций.
\end{enumerate}

\begin{quotation}
  \emph{<<Понятно, что метод сходится>>.}
  \flushbottom --- Киреева.
\end{quotation}
\end{alg}

\begin{alg}[метод Ньютона, метод касательных]
  Пусть $ f(x) \in C^2[a,b] $, имеет единственную стационарную точку и $ f'(a) <
  0$, $ f'(b) > 0 $.

  Выберем начальную точку $ x_0 $.
  \begin{enumerate}[label=(\roman*)]
    \item $f'(x_0) = 0$. Тогда $ x_0 $ стационарная.
    \item $f'(x_0) \neq 0$. Тогда проведём касательную.
  \end{enumerate}
  
\end{alg}

	\paragraph{2. Метод золотого сечения.}
	\begin{definition}
		Золотым сечением отрезка $\left[x,y\right]$ называется деление отрезка на две неравные части, так чтобы отношение длины всего отрезка к длине большей части равнялось отношению длины большей части к длине меньшей части.
	\end{definition}
	\begin{utv}
		Золотое сечение отрезка производится двумя точками $u_1, u_2$, которые расположены симметрично относительно отрезка и
		\begin{align}
			u_1 &= x+\dfrac{3-\sqrt{5}}{2}(y-x)\approx x+0.382(y-x)\\
			u_2 &= x+\dfrac{\sqrt{5}-1}{2}(y-x)\approx x+0.618(y-x)
		\end{align}
	\end{utv}
	\begin{proof}
	\begin{equation}
		x\leq u_1\leq u_2\leq y
	\end{equation}
	\begin{align}
		\dfrac{y-x}{y-u_1}&=\dfrac{y-u_1}{u_1-x}\\
		(y-x)(u_1-x)&=(y-u_1)^2
	\end{align}
	Сгруппируем, чтобы было квадратное уравнение относительно $u_1$:
	\begin{equation}
		u_1^2 + (-3y+x)u_1+y^2+xy-x^2=0
	\end{equation}
	\begin{equation}
		\mathrm{D}=5(x-y)^2 \implies u_{1,2}=\dfrac{3y-x\pm\sqrt{5}(y-x)}{2}
	\end{equation}
	Корень $u_1$ не подходит, поскольку он больше $y$:
	\begin{equation}
		u_1 = y+\underset{>0}{(y-x)}\underset{>0}{\dfrac{1+\sqrt{5}}{2}} > y
	\end{equation}
	Подходит корень $u_2$:
	\begin{equation}
		u_2 = \dfrac{3y-x-\sqrt{5}(y-x)}{2} = \dots = x+(y-x)\dfrac{3+\sqrt{5}}{2}
	\end{equation}
	Аналогично, из соображений легко получить точку $u_1$.
	\end{proof}
	\begin{utv}
		Пусть точки $u_1, u_2$ производят золотое сечение отрезка $\left[x, y\right]$.
		
		Тогда точка $u_1$ производит золотое сечение отрезка $[x, u_2)$, а точка $u_2$ - отрезка $[u_1,y]$
	\end{utv}
	\paragraph{Изложим алгоритм}
	\begin{enumerate}
		\item Пусть $f(x)$ - унимодальная на $[a, b]$.
		\item Положим $a_1=a; b_1=b$. 
		\item Найдем $u_1, u_2$, образующие золотое сечение $[a, b]$. 
		\item Вычислим $f(u_1), f(u_2)$.
		\item Если $f(u_1) \leq f(u_2)$
		\begin{itemize}
			\item $a_2 = a_1$
			\item $b_2 = u_2$
			\item $\overline{u_2}=u_1$
		\end{itemize}
		\item Если $f(u_1) > f(u_2)$
		\begin{itemize}
			\item $a_2 = u_1$
			\item $b_2 = b_1$
			\item $\overline{u_2} = u_2$
		\end{itemize}
		\item Полагаем $f(x)$ - унимодальная на $[a_2, b_2]$.
		\item $\Omega \cap [a_2, b_2] \neq \emptyset$, а также $l([a_2, b_2])=\dfrac{\sqrt{5}-1}{2}(b-a)<l([a_1, b_1])$ - длина уменьшилась
		\item[$\vdots$] и т.д.
		\item[N-ый шаг.]
	\begin{itemize}
		\item $[a_{n-1}, b_{n-1}], \quad \Omega \cap [a_{n-1}, b_{n-1}] \neq \emptyset$
		\item $l([a_{n-1}, b_{n-1}]) = \left(\dfrac{\sqrt{5}-1}{2}\right)^n(b-a)$
		\item Определим точки $u_1, u_2,\dots,u_{n-1}$
		\item Вычислим $f(u_1), f(u_2),\dots,f(u_{n-1})$ и имеется $\overline{u}_{n-1}$.
		\item Возьмем точку $u_n: u_n\neq \overline{u}_{n-1}$ и произведем золотое сечение отрезка $[a_{n-1}, b_{n-1}]$
		\item Вычислим $f(u_n)$
		\item Если $f(u_n)\leq f(\overline{u}_{n-1})$
		\begin{itemize}
			\item $a_n=a_{n-1}$
			\item $b_n = \overline{u}_{n-1}$
			\item $\overline{u}_n=u_n$
		\end{itemize}
		
		\item Если $f(u_n) > f(\overline{u}_{n-1})$
		\begin{itemize}
			\item $a_n=u_n$
			\item $b_n = b_{n-1}$
			\item $\overline{u}_n=\overline{u}_{n-1}$
		\end{itemize}
		\item $a_{n-1}<u_n<\overline{u}_{n-1}<b_{n-1}$
	\end{itemize}
