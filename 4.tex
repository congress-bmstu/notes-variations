\section{Одномерная оптимизация с использованием дифференцирования}
Пусть функция дифференцируема и мы знаем её производную. Однако по какой-то
причине уравнение $ f'(x) = 0 $ мы решить не в состоянии.
\subsection{Метод средней точки}

Пусть функция непрерывно дифференцируемая на $ [a,b] $ и имеет единственную
стационарную точку (где $ f'(x) = 0 $). Пусть также $ f'(a) <0 $, $ f'(b) > 0
$\footnote{Производная известна в том смысле, что мы можем прикинуть её знак в
точках.}. Тогда стационарная точка --- точка минимума (очевидно).

Положим $ a_0 = a $, $ b_0 = b $. Возьмём сеередину отрезка $ [a, b] $,  
\[
    c_0 = \frac{a+b}{2}.
\]
Вычислим $ f'(c_0) $. Далее как обычно в методе отрезке пополам.


\begin{alg}[Метод хорд]
  \begin{remark*} Пусть $ g(x) $ определена в точках $ x $, $ x_1 $, $ x_2 $, $
    x_1 \neq x_2 $ и $ g(x_1)\cdot g(x_2) < 0 $. Тогда прямая, проходящая через
    точки $ (x_1, g(x_1)) $ и $ (x_2, g(x_2)) $ имеет уравнение  
    \[
        y = \frac{g(x_2) - g(x_1)}{x_2 - x_1} x + \frac{g(x_1)x_2 -
        g(x_2)x_1}{x_2 - x_1},
    \]
   а точка пересечения этой прямой с осью абсцисс  
   \[
       x = \frac{g(x_2)x_1 - g(x_1)x_2}{g(x_2) - g(x_1)} = x_1 -
       \frac{g(x_1)(x_2-x_1)}{g(x_2) - g(x_1)} \in (x_1, x_2).
   \]
 \end{remark*}

 Пусть $ f(x) $ непрерывно дифференцируема на $ [a,b] $, имеет единственную
 стационарную точку на $ [a, b] $; $ f'(a) < 0 $, $ f'(b) > 0 $.

 Положим $ a_0 = a $, $ b_0 = b $.

 \texttt{$ (k+1) $-й шаг.} Рассматривается отрезок $ [a_k, b_k] $, $ f'(a_k) < 0
 $, $ f'(b_k) > 0 $. 

 Проведём прямую через точки $ (a_k, f'(a_k)) $, $ (b_k, f'(b_k)) $ и обозначим
 через 
 \[
     c_k = \frac{f'(b_k)a_k - f'(a_k)b_k}{f'(b_k) - f'(a_k)} \in (a_k, b_k).
 \]
Вычислим $ f'(c_k) $.
\begin{enumerate}[label=(\roman*).]
  \item $f'(c_k) = 0$. Тогда $ c_k  $ --- стационарная точка.
  \item $ f'(c_k) > 0 $. Тогда $ a_{k+1} = a_k $, $ b_{k+1} = c_k $.
  \item $ f'(c_k) < 0 $. Тогда $ a_{k+1} = c_k $, $ b_{k+1} = b_k $.
    %TODO: дописать
\end{enumerate}

\texttt{Условие остановки алгоритма.}
\begin{enumerate}[label=(\alph*)]
  \item Нашли стационарную точку (вряд ли),
  \item $ l([a_k, b_k]) < \varepsilon > 0$, где $ \varepsilon $ задан заранееб
  \item Выполнено заданное количество итераций.
\end{enumerate}

\begin{quotation}
  \emph{<<Понятно, что метод сходится>>.}
  \flushbottom --- Киреева.
\end{quotation}
\end{alg}

\begin{alg}[метод Ньютона, метод касательных]
  Пусть $ f(x) \in C^2[a,b] $, имеет единственную стационарную точку и $ f'(a) <
  0$, $ f'(b) > 0 $.

  Выберем начальную точку $ x_0 $.
  \begin{enumerate}[label=(\roman*)]
    \item $f'(x_0) = 0$. Тогда $ x_0 $ стационарная.
    \item $f'(x_0) \neq 0$. Тогда проведём касательную.
  \end{enumerate}
  
\end{alg}
